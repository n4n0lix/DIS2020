\documentclass[a4paper,english,abstract=on]{scrartcl}

\usepackage{mathtools} % loads amsmath and fixes its bugs in Unicode & XeLaTeX/LuaLaTeX 
\usepackage[english]{babel}
\usepackage[]{unicode-math} % provides Unicode Math support for XeLaTeX/LuaLaTeX 
\usepackage{xcolor}
\usepackage{graphicx}
\usepackage[pdfborder={0 0 0}]{hyperref}
\usepackage[autostyle=true]{csquotes}
\usepackage[backend=biber, style=numeric-comp]{biblatex}
\addbibresource{literatur.bib}
\usepackage{listings}
\lstset{language=SQL}

\title{Exercise 4 Report}
\subtitle{Gruppe 16}
\author{Anastasiia Rubanovych\and Sebastian Funck}
\date{\today}

\begin{document}

\maketitle

\subsection*{a)}
\begin{itemize}
	\item \textbf{Describe in your own words how the program works and how the persistence manager accom-
		plishes logging and recovery.}\\
	
	The persistance manager provides the interface to create transactions that modify an underlying data struture consisting of pages and user data. It is implemented as a Singleton so it has sole control over write and read operations on the respective pages and their content. For the user not visible are also a Logging and Recovery mechanism. 
	
	Logging is realized via a \texttt{Logger} inside the \texttt{PersitenceManager} that encapsulates log-writing operations. Before any transaction operation (begin, write or commit), the \texttt{PersitenceManager} ensures that this operation is written to a log, to enable recovery possibilities in case of a failure.
	
	The recovery is handled by the  \texttt{PersitenceManager} directly and is started on startup. TODO
\end{itemize}

\subsection*{b)}
\begin{lstlisting}
create table OPK (
  ID   int4,
  NAME varchar(64)
)
\end{lstlisting}

\subsection*{c)}
\begin{lstlisting}
insert into OPK (ID, NAME) 	
    VALUES (1, 'shaggy'), (2, 'fred'),
    (3, 'velma'), (4, 'scooby'), (5, 'daphne');
\end{lstlisting}

\subsection*{d)}
\begin{itemize}
\item \textbf{Discuss what locks you would expect are held at this point (and before a commit) with \texttt{Read Committed}?}\\
We would expect 1 shared table lock and 1 row lock to be hold before and 0 table locks and 0 row locks to be hold after the commit. \texttt{Cursor Stability (Read Committed)} never holds more than 1 row lock.

\item \textbf{Discuss what locks you would expect are held at this point (and before a commit) with \texttt{Repeatable Read}?}\\
We would expect 1 table lock to be hold before and 0 locks after the commit. \texttt{Repeatable Read} locks the complete table until commit.
\end{itemize}

\section*{3.2 Lock Conflicts}
\subsection*{a)}
\begin{itemize}
	\item \textbf{What happens? What is the output of Connection 1?}\\
	While Connection 1 is executing the query, Connection 2 inserts a row that matches the query of Connection 1. This results in Connection 1 commiting after Connection 2, but will not select the newly inserted item. Therefor the output of Connection 1 is
	\begin{lstlisting}
	(4, 'scooby')
	(5, 'daphne')
	\end{lstlisting}
	\item \textbf{Compare the state before the transactions with the state after the transactions.}\\
	After the commit of Connection 2 one new row has been added to the OPK table. This happens while Connection 1 is still executing.
	\item \textbf{What can be observed if Connection 1 commits and execute its SQL command again?}\\
	The output of Connection 1 becomes
	\begin{lstlisting}
	(4, 'scooby')
	(5, 'daphne')
	(6, 'scrappy')
	\end{lstlisting}
	\item \textbf{Can we observe a Canonical Synchronization Problem? If yes, explain which one and why it appears.}\\
	Yes, we can observe a Phantom Read because Connection 1 is querying a range while Connection 2 inserts a record \texttt{(6, `scrappy`}) into the table which would match that queried range.
\end{itemize}

\subsection*{b)}
	\begin{itemize}
		\item \textbf{What happens? What is the output of Connection 1?}\\
	Because Connection 1 is set to isolation mode \texttt{Serializable} the scheduler has to ensure that this transactions is executed in a serializable linear order. Therefor Connection 2 has to wait for Transaction 1 to finish. 
	\begin{lstlisting}
	(4, 'scooby')
	(5, 'daphne')
	\end{lstlisting}
	\item \textbf{Compare the state before the transactions with the state after the transactions.}\\
	After the transaction of Connection 1 there are only 5 rows in the data set. Only after connection 2 has been completed, the 6th row is added.
	\item \textbf{What can be observed if Connection 1 commits and execute its SQL command again?}\\
	The output of Connection 1 stays the same
	\begin{lstlisting}
	(4, 'scooby')
	(5, 'daphne')
	(6, 'scrappy')
	\end{lstlisting}
	\item \textbf{Can we observe a Canonical Synchronization Problem? If yes, explain which one and why it appears.}\\
	By definition of isolation level \texttt{Serializable} there can't occour any canonical synchronization problems.
\end{itemize}

\subsection*{c)}
\begin{itemize}
	\item \textbf{In this scenario Connection 2 has to wait until Connection 1 commits. Explain why.}\\
	\texttt{ID=1} is in the context of table \texttt{OKP} a range based query, because multiple rows could match this criteria. In addition Connection 1 is in isolation level \texttt{Serializable}. This results in Connection 1 holding an R-lock on the complete table \texttt{OKP} and Connection 2 waiting for Connection 1 to commit.
	\item \textbf{Discuss, what lock can be potentially encountered on the table OPK? Which Connection do the locks belong to?}\\
	Connection 1 is likely to hold an R lock on the table \texttt{OKP}.
\end{itemize}

\subsection*{d)}
\begin{lstlisting}
create table MPK (
	id int4,
	name varchar(64),
	CONSTRAINT mpk_pk PRIMARY KEY (id)
);

insert into MPK (ID, NAME) 
	VALUES (1, 'shaggy'), (2, 'fred'), 
	(3, 'velma'), (4, 'scooby'), (5, 'daphne');
\end{lstlisting}

\subsection*{e)}
With \texttt{ID} being a \texttt{PRIMARY KEY}, every row is uniqly identifiable by its id. This means the query of Connection 1 is no longer range-query and enables Connection 1 to hold a narrower lock (R-lock on row \texttt{ID:3}). On the otherhand Connection 2 now can hold an R-lock on row \texttt{ID:3} while Connection 1 is executing and doesn't have to wait for it to finish.

\end{document}
