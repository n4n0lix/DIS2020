\documentclass[a4paper,english,abstract=on]{scrartcl}

\usepackage{mathtools} % loads amsmath and fixes its bugs in Unicode & XeLaTeX/LuaLaTeX 
\usepackage[english]{babel}
\usepackage[]{unicode-math} % provides Unicode Math support for XeLaTeX/LuaLaTeX 
\usepackage{xcolor}
\usepackage{graphicx}
\usepackage[pdfborder={0 0 0}]{hyperref}
\usepackage[autostyle=true]{csquotes}
\usepackage[backend=biber, style=numeric-comp]{biblatex}
\usepackage{listings}
\lstset{language=Bash}

\usepackage{fontspec}
\newfontfamily{\ttconsolas}{Consolas}
\usepackage{listings}
\setmonofont{Consolas}
\lstset{
	breaklines=true,
	tabsize=2,
	basicstyle=\ttfamily\tiny,
}

\addbibresource{literatur.bib}

\title{Exercise 7 Report}
\subtitle{Gruppe 16}
\author{Anastasiia Rubanovych\and Sebastian Funck}
\date{\today}

\begin{document}

\maketitle

\textbf{Briefy describe how the algorithm and your software work.}
~\\~\\
The algorithm incrementally checks the dataset for subsets of transactions that fulfill a minimum support. The support of a set denotes the frequency that the set appears in transactions. Instead of checking every possible subset, the algorithm starts with finding 1-item sets that fullfil the minimun support and expands on those. That means out of the $k-1$-itemsets that meet the minimum support we construct all possible $k$-itemsets and repeat the process. Finally we get a mapping from subset to it's support value. The higher the support-value the higher the amount of transactions that are containing these items.
~\\~\\
\textbf{List how many frequent itemsets you obtained containing \{one,two,three,...\} items.}
\begin{itemize}
	\item 1-item: 379
	\item 2-item: 11
	\item 3-item: 1
\end{itemize}

\newpage
\textbf{Provide the found frequent itemsets containing two or more items together with their computed support value.}
\begin{lstlisting}
[39, 704] => 0.0139
[39, 825] => 0.0151
[217, 283] => 0.0104
[217, 346] => 0.0149
[227, 390] => 0.0101
[368, 682] => 0.0128
[368, 692] => 0.0114
[368, 829] => 0.0127
[390, 722] => 0.0103
[704, 825] => 0.0135
[789, 829] => 0.0119
[39, 704, 825] => 0.0128
\end{lstlisting}
\end{document}
